% !TeX spellcheck = de_CH
\documentclass[11pt, a4paper,oneside]{scrartcl}

% header
\usepackage[headsepline]{scrlayer-scrpage}
\clearpairofpagestyles
\ohead{\today}
\ihead{GVS Meeting Protokoll}
\cfoot{\pagemark}

% language
\usepackage[T1]{fontenc}
\usepackage[utf8]{inputenc}
\usepackage[ngerman]{babel}

% tables
\usepackage{tabu}
\usepackage{booktabs}
\tabulinesep=2mm

% images
\usepackage{graphicx}

% layout
\usepackage[]{geometry}
\usepackage{multicol}
\setlength{\columnsep}{1cm}


% math
\usepackage{amsmath}
\usepackage{amssymb}
\usepackage{amsfonts}
\usepackage{enumitem}



\begin{document}
	\centering
	\rule{\textwidth}{1.6pt}\vspace*{-\baselineskip}\vspace*{2pt} %Thick horizontal line.
	\rule{\textwidth}{0.4pt}\\[\baselineskip] %Thin horizontal line.
	{\LARGE [ Meeting Protokoll Woche 12 ]}\\[0.2\baselineskip] %Title.
	\rule{\textwidth}{0.4pt}\vspace*{-\baselineskip}\vspace{3.2pt} %Thin horizontal line.
	\rule{\textwidth}{1.6pt}\\[2\baselineskip] %Thick horizontal line.
	
	\begin{tabu} to \linewidth {l X }
		\toprule
		\textbf{Thema} & \textbf{Wöchentliches GVS Meeting} \\
		\midrule
		Ort & Raum 5.207 \\
		Datum & 06.12.2017  \\
		Uhrzeit &  17:10 - 17:45 \\
		Teilnehmer & 
		\begin{minipage}[t]{\textwidth}
			\begin{itemize}
				\item Murièle Trentini
				\item Michael Wieland
				\item Thomas Letsch
			\end{itemize}
		\end{minipage}
		\\
		\bottomrule
	\end{tabu}
	
	
	\section{Rückblick}
	\begin{enumerate}
		\item Die Programmierarbeiten wurden weitgehend abgeschlossen. Ausstehend sind noch kleinere Verbesserungen sowie die Arbeiten an der \textit{ScalableScrollPane} Komponente.
		\item Das Projektteam hat Release 2 durchgeführt und an Herr Letsch gesendet.
		\item Das Projektteam hat Systemtests durchgeführt und die gefundenen Fehler als Bug Issue im Jira erfasst.
		\item Das Projektteam hat das Enterprise Architect Domainmodell mit dem aktuellen Programmcode synchronisiert.
	\end{enumerate}
	
	\section{Aktuelles}
		\begin{enumerate}
		\item Die \textit{ScalableScrollPane} verhält sich für einige Graphen noch nicht ordnungsgemäss. Das Projektteam arbeitet an einer verbesserten Variante.
	\end{enumerate} 
	
	\section{Beschlüsse}
	\begin{enumerate}
		\item Die .NET Lib muss nicht weiter getestet werden.
		\item Der Server Startport soll weiter konfigurierbar bleiben.
		\item Im Enterprise Architect müssen keine zusätzliche Relationen nachgetragen werden. Allenfalls können diese in einem kopierten Modell für den Projektplan eingezeichnet werden.
		\item Für Mixed Graphen (Relativ und Deafault Vertices) sollen verschiedenen Lösungsansätze getestet und der einfachste umgesetzt werden. Aktuell wird im Server eine Exception geworfen, da das XML Schema solche Graphen nicht erkennt.
		\item Wenn der Benutzer Trees ohne Root überträgt, wird aktuell nichts angezeigt. Auf eine Meldung kann verzichtet werden, da der Benutzer selber verantwortlich ist, was er an den Server sendet.
		\item Weitere Arbeiten (wie z.B das Refactoring von ModelBuilder und Persistor) die nicht mehr in den Scope gepasst haben, sollen in der Projektdokumentation beschrieben werden.
		\item Das Projektteam hat die Nutzungsvereinbarung von Herr Letsch erhalten und wird eine eingescannte Version der Projektdokumentation hinzufügen.
		\item Die nächsten beiden Meetings werden am Dienstag stattfinden.
	\end{enumerate}

	\section{Ausblick}
	\begin{enumerate}
		\item In der nächsten Woche wird die Projektdokumentation aktualisiert. 
		\item Ebenfalls werden noch kleinere Bugs aus den Systemtests behoben.
	\end{enumerate}
	
	\section{Nächster Termin}
	\begin{tabu} to \linewidth {l X }
		\toprule
		Termin & 12.12.2017  \\
		Bemerkungen & Ort wird noch bekannt gegeben  \\
		\bottomrule
	\end{tabu}
	
\end{document}
