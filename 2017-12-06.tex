% !TeX spellcheck = de_CH
\documentclass[11pt, a4paper,oneside]{scrartcl}

% header
\usepackage[headsepline]{scrlayer-scrpage}
\clearpairofpagestyles
\ohead{\today}
\ihead{GVS Meeting Protokoll}
\cfoot{\pagemark}

% language
\usepackage[T1]{fontenc}
\usepackage[utf8]{inputenc}
\usepackage[ngerman]{babel}

% tables
\usepackage{tabu}
\usepackage{booktabs}
\tabulinesep=2mm

% images
\usepackage{graphicx}

% layout
\usepackage[]{geometry}
\usepackage{multicol}
\setlength{\columnsep}{1cm}


% math
\usepackage{amsmath}
\usepackage{amssymb}
\usepackage{amsfonts}
\usepackage{enumitem}



\begin{document}
	\centering
	\rule{\textwidth}{1.6pt}\vspace*{-\baselineskip}\vspace*{2pt} %Thick horizontal line.
	\rule{\textwidth}{0.4pt}\\[\baselineskip] %Thin horizontal line.
	{\LARGE [ Meeting Protokoll Woche 12 ]}\\[0.2\baselineskip] %Title.
	\rule{\textwidth}{0.4pt}\vspace*{-\baselineskip}\vspace{3.2pt} %Thin horizontal line.
	\rule{\textwidth}{1.6pt}\\[2\baselineskip] %Thick horizontal line.
	
	\begin{tabu} to \linewidth {l X }
		\toprule
		\textbf{Thema} & \textbf{Wöchentliches GVS Meeting} \\
		\midrule
		Ort & Raum 1.223 \\
		Datum & 06.12.2017  \\
		Uhrzeit &  17:10 -  \\
		Teilnehmer & 
		\begin{minipage}[t]{\textwidth}
			\begin{itemize}
				\item Murièle Trentini
				\item Michael Wieland
				\item Thomas Letsch
			\end{itemize}
		\end{minipage}
		\\
		\bottomrule
	\end{tabu}
	
	
	\section{Rückblick}
	\begin{enumerate}
		\item Die Programmierarbeiten wurden weitgehend abgeschlossen. Ausstehend sind noch kleinere Verbesserungen sowie die Arbeiten an der \textit{ScalableScrollPane} Komponente.
		\item Das Projektteam hat Release 2 durchgeführt und an Herr Letsch gesendet.
		\item Das Projektteam hat Systemtests durchgeführt und die gefundenen Fehler als Bug Issue im Jira erfasst.
		\item Das Projektteam hat das Enterprise Architect Domainmodell mit dem aktuellen Programmcode synchronisisert
	\end{enumerate}
	
	\section{Aktuelles}
		\begin{enumerate}
		\item Das Verhalten der \textit{ScalableScrollPane} verhält sich für einige Graphen noch nicht ordnungsgemäss. Das Projektteam arbeitet an einer verbesserten Variante.
	\end{enumerate} 
	
	\section{Beschlüsse}
	\begin{enumerate}
		\item Wie sollen wir den .NET Client testen? Müssen Tester geschrieben werden?
		\item Ist der Konfigurierbare Startport nötig. Es wird so oder so hochgezählt?
		\item Müssen wir im Enterprise Architect zusätzliche Pfeile zeichnen? (z.B SessionFactory, Session -> Wäre für das Verständnis besser)
		\item Mixed Graph (Relativ und Deafault) Verhalten?
		\item Empty Graph: Brachen wir ein Info 	Message Feature?
		\item Weitere Arbeiten (wie z.B das Refactoring von ModelBuilder und Persistor) die nicht mehr in den Scope gepasst haben, sollen in der Projektdokumentation beschrieben werden.
	\end{enumerate}

	\section{Ausblick}
	\begin{enumerate}
		\item In der nächsten Woche wird die Projektdokumentation aktualisiert. 
		\item Ebenfalls werden die noch kleinere Bugs aus den Systemtests behoben.
	\end{enumerate}
	
	\section{Nächster Termin}
	\begin{tabu} to \linewidth {l X }
		\toprule
		Termin & 13.12.2017  \\
		Bemerkungen & Ort wird noch bekannt gegeben  \\
		\bottomrule
	\end{tabu}
	
\end{document}
