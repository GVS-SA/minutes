% !TeX spellcheck = de_CH
\documentclass[11pt, a4paper,oneside]{scrartcl}

% header
\usepackage[headsepline]{scrlayer-scrpage}
\clearpairofpagestyles
\ohead{\today}
\ihead{GVS Meeting Protokoll}
\cfoot{\pagemark}

% language
\usepackage[T1]{fontenc}
\usepackage[utf8]{inputenc}
\usepackage[ngerman]{babel}

% tables
\usepackage{tabu}
\usepackage{booktabs}
\tabulinesep=2mm

% images
\usepackage{graphicx}

% layout
\usepackage[]{geometry}
\usepackage{multicol}
\setlength{\columnsep}{1cm}


% math
\usepackage{amsmath}
\usepackage{amssymb}
\usepackage{amsfonts}
\usepackage{enumitem}



\begin{document}
	\centering
	\rule{\textwidth}{1.6pt}\vspace*{-\baselineskip}\vspace*{2pt} %Thick horizontal line.
	\rule{\textwidth}{0.4pt}\\[\baselineskip] %Thin horizontal line.
	{\LARGE [ Meeting Protokoll Woche 14 ]}\\[0.2\baselineskip] %Title.
	\rule{\textwidth}{0.4pt}\vspace*{-\baselineskip}\vspace{3.2pt} %Thin horizontal line.
	\rule{\textwidth}{1.6pt}\\[2\baselineskip] %Thick horizontal line.
	
	\begin{tabu} to \linewidth {l X }
		\toprule
		\textbf{Thema} & \textbf{Wöchentliches GVS Meeting} \\
		\midrule
		Ort & Raum 1.223 \\
		Datum & 19.12.2017  \\
		Uhrzeit &  17:10 - 17:50  \\
		Teilnehmer & 
		\begin{minipage}[t]{\textwidth}
			\begin{itemize}
				\item Murièle Trentini
				\item Michael Wieland
				\item Thomas Letsch
			\end{itemize}
		\end{minipage}
		\\
		\bottomrule
	\end{tabu}
	
	
	\section{Rückblick}
	\begin{enumerate}
		\item Das Projektteam hat in den letzten 14 Wochen einen funktionsfähigen Ersatz für den GVS 1.0 entwickelt.
		\item GVS 2.0 unterstützt nun auch die Darstellung von n-ary Trees. Das Feature wurde noch in den Umfang aufgenommen, obschon die Darstellung noch nicht ganz fehlerfrei funktioniert. 
	\end{enumerate}
	
	\section{Aktuelles}
	\begin{enumerate}
		\item Abstract und Poster wurde zum Review übergeben
	\end{enumerate} 
	
	\section{Beschlüsse}
	\begin{enumerate}
		\item Bei der Zeitauswertung genügt eine Übersicht über alle Kategorien sowie pro Teammitglied. Auf einen Auszug aller Arbeitspakete darf verzichtet werden. 
		\item Das Projektteam hat nützliches Feedback zum Abstract und Poster erhalten und wird dieses in den nächsten Tagen noch umsetzen.
		\item Die verwendeten Versionen der IDE's, Plugins und Frameworks werden noch dokumentiert.
		\item Folgende Dokumente werden auf der DVD enthalten sein:
		\begin{enumerate}
			\item Sourcecode UI Komponente
			\item Sourcecode Lib (Java und C\#)
			\item Sourcecode Testers
			\item Enterprise Architect Project (inkl. aktuelles Klassendiagramm)
			\item Benutzerhandbuch (separat)
			\item Projektdokumentation mit Anhängen (Abstract, Management Summary, Anforderungsspezifikation, Architektur und Designspezifikation, Umsetzung, Ergebnisdiskussion, Projektplanung, Meetingprotokolle, Zeitauswertung, Testprotokoll, Persönliche Berichte, Eigenständigkeitserklärung)
			\item Poster (A0)
		\end{enumerate}
	\end{enumerate}
	
	\section{Ausblick}
	\begin{enumerate}
		\item Das Projektteam freut sich auf eine spannende Bachelorarbeit und bedankt sich für die gute Zusammenarbeit.		
	\end{enumerate}
	
	\section{Nächster Termin}
	\begin{tabu} to \linewidth {l X }
		\toprule
		Termin & -  \\
		Bemerkungen &  \\
		\bottomrule
	\end{tabu}
	
\end{document}
