% !TeX spellcheck = de_CH
\documentclass[11pt, a4paper,oneside]{scrartcl}

% header
\usepackage[headsepline]{scrlayer-scrpage}
\clearpairofpagestyles
\ohead{\today}
\ihead{GVS Meeting Protokoll}
\cfoot{\pagemark}

% language
\usepackage[T1]{fontenc}
\usepackage[utf8]{inputenc}
\usepackage[ngerman]{babel}

% tables
\usepackage{tabu}
\usepackage{booktabs}
\tabulinesep=2mm

% images
\usepackage{graphicx}

% layout
\usepackage[]{geometry}
\usepackage{multicol}
\setlength{\columnsep}{1cm}


% math
\usepackage{amsmath}
\usepackage{amssymb}
\usepackage{amsfonts}
\usepackage{enumitem}



\begin{document}
\centering
\rule{\textwidth}{1.6pt}\vspace*{-\baselineskip}\vspace*{2pt} %Thick horizontal line.
\rule{\textwidth}{0.4pt}\\[\baselineskip] %Thin horizontal line.
{\LARGE [ Meeting Protokoll Woche 2 ]}\\[0.2\baselineskip] %Title.
\rule{\textwidth}{0.4pt}\vspace*{-\baselineskip}\vspace{3.2pt} %Thin horizontal line.
\rule{\textwidth}{1.6pt}\\[2\baselineskip] %Thick horizontal line.

\begin{tabu} to \linewidth {l X }
	\toprule
	\textbf{Thema} & \textbf{Wöchentliches GVS Meeting} \\
	\midrule
	Ort & Raum 5.207 (HSR) \\
	Datum & 27.09.2017 \\
	Uhrzeit & 17:15 - 18:15 \\
	Teilnehmer & 
	\begin{minipage}[t]{\textwidth}
	  	\begin{itemize}
	  		\item Thomas Letsch
			\item Murièle Trentini
			\item Michael Wieland
	  	\end{itemize}
	\end{minipage}
	\\
	\bottomrule
\end{tabu}


\section{Rückblick}
\begin{enumerate}
	\item Der erste Sprint wurde erfolgreich abgeschlossen 
	\item Das Projektteam hat den Abschnitt "Projektmanagement" dokumentiert und die Artefakt-Übersicht wie im ersten Meeting besprochen, überarbeitet.
	\item Das Projektteam erarbeitete sich eine Überblick über den GVS V1. Aufgrund den Erkenntnissen sollte ein Refactoring der Visualization Komponente möglich sein. 
	\item Das Projektteam informierte sich über JavaFX und erstellte einen minimalen Prototyp um erste Erfahrungen mit der Technologie zu sammeln
\end{enumerate}

\section{Aktuelles}
\begin{enumerate}
	\item Die Klasse "javafx.embed.swing.SwingNode" könnte während der Entwicklung verwendet werden, um noch nicht migrierte Swing Komponenten trotzdem anzuzeigen.
	\item Als eine weitere Variante wird in Betracht gezogen, neben dem aktuellen Swing GUI ein separates JavaFX GUI zu entwickeln, welches Parallel als Observer auf das Model zugreift. Dies gewährt einen 1:1 Vergleich des neuen Codes mit dem bisherigen.
\end{enumerate}

\section{Beschlüsse}
\begin{enumerate}
	\item Die Meeting Protokolle sollen wenn möglich so generiert werden, dass alle Unicode Zeichen separat dargestellt werden. Der Latex Builder ist entsprechend einzurichten. 
	\item Der Projektbericht soll wenn möglich in mehrere Dokumente unterteilt werden und zum Schluss des Projekts zusammengeführt werden. Ist dies ohne grösseren Aufwand möglich, soll pro Dokument eine Versionskontrolle (Tabelle innerhalb des Dokuments) gepflegt werden.
\end{enumerate}

\section{Ausblick}
\begin{enumerate}
	\item Das Projektteam erhält die Liste mit zusätzlichen Anforderungen noch in digitaler Form. Die Punkte sollen zu einem späteren Zeitpunkt in dem ''Requirements'' Abschnitt aufgenommen werden. Alle Punkte dürfen mit ''LOW'' bewertet werden. 
	\item In dem nächsten Sprint soll ein erster Prototyp erstellt werden um die Wiederverwendbarkeit von GVS V1 am realen Beispiel zu prüfen. 
	\item Das Projektteam erhält ein ZIP File mit der neusten GVS V1.5 Version inklusive einem generierten Klassendiagramm für den Enterprise Architect. 
\end{enumerate}

\section{Nächster Termin}
\begin{tabu} to \linewidth {l X }
	\toprule
	Termin & 04.10.2017 - 17:15 \\
	Ort & Raum 5.207 (HSR) \\
	Bemerkungen & - \\
	\bottomrule
\end{tabu}

\end{document}