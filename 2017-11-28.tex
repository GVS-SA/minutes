% !TeX spellcheck = de_CH
\documentclass[11pt, a4paper,oneside]{scrartcl}

% header
\usepackage[headsepline]{scrlayer-scrpage}
\clearpairofpagestyles
\ohead{\today}
\ihead{GVS Meeting Protokoll}
\cfoot{\pagemark}

% language
\usepackage[T1]{fontenc}
\usepackage[utf8]{inputenc}
\usepackage[ngerman]{babel}

% tables
\usepackage{tabu}
\usepackage{booktabs}
\tabulinesep=2mm

% images
\usepackage{graphicx}

% layout
\usepackage[]{geometry}
\usepackage{multicol}
\setlength{\columnsep}{1cm}


% math
\usepackage{amsmath}
\usepackage{amssymb}
\usepackage{amsfonts}
\usepackage{enumitem}



\begin{document}
	\centering
	\rule{\textwidth}{1.6pt}\vspace*{-\baselineskip}\vspace*{2pt} %Thick horizontal line.
	\rule{\textwidth}{0.4pt}\\[\baselineskip] %Thin horizontal line.
	{\LARGE [ Meeting Protokoll Woche 11 ]}\\[0.2\baselineskip] %Title.
	\rule{\textwidth}{0.4pt}\vspace*{-\baselineskip}\vspace{3.2pt} %Thin horizontal line.
	\rule{\textwidth}{1.6pt}\\[2\baselineskip] %Thick horizontal line.
	
	\begin{tabu} to \linewidth {l X }
		\toprule
		\textbf{Thema} & \textbf{Wöchentliches GVS Meeting} \\
		\midrule
		Ort & Raum 1.223 \\
		Datum & 28.11.2017  \\
		Uhrzeit &  17:10 - 18:10 \\
		Teilnehmer & 
		\begin{minipage}[t]{\textwidth}
			\begin{itemize}
				\item Murièle Trentini
				\item Michael Wieland
				\item Thomas Letsch
			\end{itemize}
		\end{minipage}
		\\
		\bottomrule
	\end{tabu}
	
	
	\section{Rückblick}
	\begin{enumerate}
		\item Der TreeLayouter wurde verbessert (z.B. werden Links- und Rechtsbäume nun korrekt dargestellt) und macht den Cluster Splitter nun redundant 
		\item Für die Java und .NET Libraries wurden die Entwicklungsumgebungen inkl. Git Repository aufgesetzt.
		\item Die Java- und .NET-Library wurden an die aktuellen Ansprüche angepasst
		\begin{enumerate}
			\item Styles wurden an GVS 2.0 UI angepasst
			\item Felder \textit{Background} und \textit{MaxLabelLength} sind nun fakultativ und werden vom GVS 2.0 UI nicht beachtet, falls sie mitgeschickt werden.
			\item Generics wurden für die Klassenvariablen eingeführt. Auf der Schnittstelle sind keine Generics nötig.
		\end{enumerate}
	\item Diverse BugFixes (u.a. gemäss Feedback aus Release 1)
	\end{enumerate}
	
	\section{Aktuelles}
		\begin{enumerate}
		\item ''Komische Lücke'' in Tree mit Test File "ClusterSplitterTest" ist nun behoben.
		\item Test Files wurden aus dem Java Lib Project entfernt und in ein eigenständiges Java Projekt verschoben.
	\end{enumerate} 
	
	\section{Beschlüsse}
	\begin{enumerate}
		\item Die \textit{maxLabelLength} wird von GVS 2.0 nicht mehr unterstützt (optional auf dem Interface). Als Ersatz werden zu lange Labels in der Mitte durch Punkte ersetzt. (''TestLabel12'' $\Rightarrow$ ''Te...12'').
		\item Generics sind auf dem Interface des Java und .NET Clients nicht notwendig.
		\item Die Projektdokumentation muss vom Projektteam nicht ausgedruckt werden und darf digital abgegeben werden.	
		\item In der Projektdokumentation sollen alle Änderungen zum GVS 1.0 dokumentiert werden. (z.B \textit{maxLabelLength} wurde entfernt)
		\item Das Projektteam erhält von Herr Letsch allfällige TestFiles für die .NET Library. (falls vorhanden)
		\item Das Projektteam erhält von Herr Letsch eine aktualisierte Anleitung für die Synchronisation im Enterprise Architect.
	\end{enumerate} 
	
	\section{Ausblick}
	\begin{enumerate}
		\item Durchführung von Systemtests (inkl. Testprotokoll, ''Integrationstests'' in gvs-tester Repository erweitern, Bugliste erweitern und beheben)
		\item Vorbereitung von Release 2 am Mittwoch 06.12.2017: Aktualisieren des Enterprise Architect Domainmodell gemäss Anleitung von Herr Letsch.
		\item Bei vorhandener Zeit sollen die Klassen \textit{ModelBuilder} und \textit{Persistor} zusammengelegt werden. Gemeinsame Funktionalität soll in einer neuen Klasse gekapselt werden. Ziel ist, kein duplicated Code mehr zu haben.
	\end{enumerate}
	
	\section{Nächster Termin}
	\begin{tabu} to \linewidth {l X }
		\toprule
		Termin & 06.12.2017  \\
		Bemerkungen & Ort wird noch bekannt gegeben  \\
		\bottomrule
	\end{tabu}
	
\end{document}
