% !TeX spellcheck = de_CH
\documentclass[11pt, a4paper,oneside]{scrartcl}

% header
\usepackage[headsepline]{scrlayer-scrpage}
\clearpairofpagestyles
\ohead{\today}
\ihead{GVS Meeting Protokoll}
\cfoot{\pagemark}

% language
\usepackage[T1]{fontenc}
\usepackage[utf8]{inputenc}
\usepackage[ngerman]{babel}

% tables
\usepackage{tabu}
\usepackage{booktabs}
\tabulinesep=2mm

% images
\usepackage{graphicx}

% layout
\usepackage[]{geometry}
\usepackage{multicol}
\setlength{\columnsep}{1cm}


% math
\usepackage{amsmath}
\usepackage{amssymb}
\usepackage{amsfonts}
\usepackage{enumitem}



\begin{document}
\centering
\rule{\textwidth}{1.6pt}\vspace*{-\baselineskip}\vspace*{2pt} %Thick horizontal line.
\rule{\textwidth}{0.4pt}\\[\baselineskip] %Thin horizontal line.
{\LARGE [ Meeting Protokoll Woche 1 ]}\\[0.2\baselineskip] %Title.
\rule{\textwidth}{0.4pt}\vspace*{-\baselineskip}\vspace{3.2pt} %Thin horizontal line.
\rule{\textwidth}{1.6pt}\\[2\baselineskip] %Thick horizontal line.

\begin{tabu} to \linewidth {l X }
	\toprule
	\textbf{Thema} & \textbf{Wöchentliches GVS Meeting} \\
	\midrule
	Ort & Raum 5.207 (HSR) \\
	Datum & 20.09.2017 \\
	Uhrzeit & 17:15 - 18:30 \\
	Teilnehmer & 
	\begin{minipage}[t]{\textwidth}
	  	\begin{itemize}
	  		\item Thomas Letsch
			\item Murièle Trentini
			\item Michael Wieland
	  	\end{itemize}
	\end{minipage}
	\\
	\bottomrule
\end{tabu}


\section{Rückblick}
\begin{enumerate}
	\item Es haben vorhergehenden Meetings stattgefunden
\end{enumerate}

\section{Aktuelles}
\begin{enumerate}
	\item Das Projektteam ist motiviert ein funktionales Produkt in den nächsten 14 Wochen zu entwickeln
\end{enumerate}

\section{Beschlüsse}
\begin{enumerate}
	\item Alle Entscheidungen werden in den vorliegenden Meeting Protokollen niedergeschrieben und innert 24h an alle Teilnehmer versendet.
	\item Der Entwicklungsprozess soll von Projektteam sinnvoll definiert werden. Ebenfalls sollen die Schritte so dokumentiert werden, dass sich ein neuer Betreuer rasch in das Projekt einarbeiten könnte.
	\item Die Zeiterfassung soll in sinnvolle Kategorien unterteilt werden. Dies erlaubt eine bessere Auswertung. Als Kategorien können z.B Administratives, Dokumentation, Implementierung verwendet werden.
	\item Das Artefakt-Dokument sollte noch um die von der HSR vorgegebenen Dokumente ergänzt werden (Siehe. Anleitung Dokumentation BA/SA)
	\item Das Artefakt-Dokument soll auch Pre-Release Versionen beinhalten (v1,v2,etc.)
	\item Die Rückwärtskompatibilität wäre wünschenswert, muss aber nicht zwingend gewährleistet werden.
	\item Die maximale Ausfallzeit darf maximal 8h betragen. Das Projektteam ist angehalten, sich entsprechend darauf vorzubereiten. (Labor PC aufsetzen, Ersatznotebook, Backupkonzept, Risikomanagement)
	\item Während dem Projekt sind drei Releases zwingend. Wenn möglich, sollte nach jedem Sprint ein lauffähiges Produkt zum Download angeboten werden. 
	\item In der Projektdokumentation ist darauf zu achten, dass keine Redundanzen entstehen. Redundanzen können mit Querverweisen minimiert werden. (Ausnahme: Technischer Bericht, Management Summary und Abstract. Diese Dokumente müssen eigenständig lesbar sein.)
	\item In dem Mail mit dem Meeting Protokoll sollen auch die Links zum Github Repository und der Jira Projektmanagement Site enthalten sein.
\end{enumerate}

\section{Ausblick}
\begin{enumerate}
	\item Ziel für die folgende Woche ist es, sich einen Überblick über die GVS v1 Software zu verschaffen. Dabei soll speziell das Layering analysiert werden. Gibt es wenig Abhängigkeiten zum eingesetzten UI-Framework (Swing), könnte evtl. der Businesslayer wiederverwendet werden.
	\item Beim nächsten Meeting erhält das Projektteam eine Liste mit zusätzlichen Anforderungen an die GVS v2 Software. (z.B Reconnect des Clients, bei einem Serverausfall)
	\item Sofern die Zeit da ist, erhält das Projektteam von Herrn Letsch einen Export aus dem Enterprise Architect CASE Tool, damit sie sich eine bessere Übersicht verschaffen können.
\end{enumerate}

\section{Nächster Termin}
\begin{tabu} to \linewidth {l X }
	\toprule
	Termin & 27.09.2017 \\
	Bemerkungen & - \\
	\bottomrule
\end{tabu}

\end{document}