% !TeX spellcheck = de_CH
\documentclass[11pt, a4paper,oneside]{scrartcl}

% header
\usepackage[headsepline]{scrlayer-scrpage}
\clearpairofpagestyles
\ohead{\today}
\ihead{GVS Meeting Protokoll}
\cfoot{\pagemark}

% language
\usepackage[T1]{fontenc}
\usepackage[utf8]{inputenc}
\usepackage[ngerman]{babel}

% tables
\usepackage{tabu}
\usepackage{booktabs}
\tabulinesep=2mm

% images
\usepackage{graphicx}

% layout
\usepackage[]{geometry}
\usepackage{multicol}
\setlength{\columnsep}{1cm}


% math
\usepackage{amsmath}
\usepackage{amssymb}
\usepackage{amsfonts}
\usepackage{enumitem}



\begin{document}
\centering
\rule{\textwidth}{1.6pt}\vspace*{-\baselineskip}\vspace*{2pt} %Thick horizontal line.
\rule{\textwidth}{0.4pt}\\[\baselineskip] %Thin horizontal line.
{\LARGE [ Meeting Protokoll Woche 6 ]}\\[0.2\baselineskip] %Title.
\rule{\textwidth}{0.4pt}\vspace*{-\baselineskip}\vspace{3.2pt} %Thin horizontal line.
\rule{\textwidth}{1.6pt}\\[2\baselineskip] %Thick horizontal line.

\begin{tabu} to \linewidth {l X }
	\toprule
	\textbf{Thema} & \textbf{Wöchentliches GVS Meeting} \\
	\midrule
	Ort & Raum 1.223 \\
	Datum & 07.11.2017  \\
	Uhrzeit &  17:15 -  \\
	Teilnehmer & 
	\begin{minipage}[t]{\textwidth}
	  	\begin{itemize}
			\item Murièle Trentini
			\item Michael Wieland
			\item Thomas Letsch
	  	\end{itemize}
	\end{minipage}
	\\
	\bottomrule
\end{tabu}


\section{Rückblick}
\begin{enumerate}
	\item Das Projektteam hat angefangen, Graphen auf dem Presentation Layer visuell darzustellen.
	\item Die Projektdokumentation wurde gemäss Feedback von Herrn Letsch angepasst. (Zeitformen: keine Vergangenheiten, Seitenzahlen: einheitliche Nummerierung, ...)
\end{enumerate}

\section{Aktuelles}
\begin{enumerate}
	\item Das Meeting ist auf den 7.11. vor verschoben (ursprünglich geplant: 8.11.).
\end{enumerate}

\section{Beschlüsse}
\begin{enumerate}
	\item 
\end{enumerate}

\section{Ausblick}
\begin{enumerate}
	\item Die Implementierung des Presentation Layers soll abgeschlossen werden. Insbesondere die Darstellung von Trees muss noch implementiert werden.
\end{enumerate}

\section{Nächster Termin}
\begin{tabu} to \linewidth {l X }
	\toprule
	Termin & 15.11.2017  \\
	Bemerkungen & Ort wird noch bekanntgegeben   \\
	\bottomrule
\end{tabu}

\end{document}
