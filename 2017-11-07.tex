% !TeX spellcheck = de_CH
\documentclass[11pt, a4paper,oneside]{scrartcl}

% header
\usepackage[headsepline]{scrlayer-scrpage}
\clearpairofpagestyles
\ohead{\today}
\ihead{GVS Meeting Protokoll}
\cfoot{\pagemark}

% language
\usepackage[T1]{fontenc}
\usepackage[utf8]{inputenc}
\usepackage[ngerman]{babel}

% tables
\usepackage{tabu}
\usepackage{booktabs}
\tabulinesep=2mm

% images
\usepackage{graphicx}

% layout
\usepackage[]{geometry}
\usepackage{multicol}
\setlength{\columnsep}{1cm}


% math
\usepackage{amsmath}
\usepackage{amssymb}
\usepackage{amsfonts}
\usepackage{enumitem}



\begin{document}
\centering
\rule{\textwidth}{1.6pt}\vspace*{-\baselineskip}\vspace*{2pt} %Thick horizontal line.
\rule{\textwidth}{0.4pt}\\[\baselineskip] %Thin horizontal line.
{\LARGE [ Meeting Protokoll Woche 6 ]}\\[0.2\baselineskip] %Title.
\rule{\textwidth}{0.4pt}\vspace*{-\baselineskip}\vspace{3.2pt} %Thin horizontal line.
\rule{\textwidth}{1.6pt}\\[2\baselineskip] %Thick horizontal line.

\begin{tabu} to \linewidth {l X }
	\toprule
	\textbf{Thema} & \textbf{Wöchentliches GVS Meeting} \\
	\midrule
	Ort & Raum 1.223 \\
	Datum & 07.11.2017  \\
	Uhrzeit &  17:15 - 17:40  \\
	Teilnehmer & 
	\begin{minipage}[t]{\textwidth}
	  	\begin{itemize}
			\item Murièle Trentini
			\item Michael Wieland
			\item Thomas Letsch
	  	\end{itemize}
	\end{minipage}
	\\
	\bottomrule
\end{tabu}


\section{Rückblick}
\begin{enumerate}
	\item Das Projektteam hat angefangen, Graphen auf dem Presentation Layer visuell darzustellen. 
	\item Die Projektdokumentation wurde gemäss Feedback von Herrn Letsch angepasst. (Zeitformen: keine Vergangenheiten, Seitenzahlen: einheitliche Nummerierung, ...)
\end{enumerate}

\section{Aktuelles}
\begin{enumerate}
	\item Das aktuelle Meeting ist auf den 7.11.2017 vor verschoben (ursprünglich geplant: 8.11.2017).
	\item Das Projektteam hat die Urheber und Nutzungsbedingungen von Herr Letsch erhalten.
\end{enumerate}

\section{Beschlüsse}
\begin{enumerate}
	\item Gemäss den offiziellen HSR Richtlinien SA/BA ist ein Logging mit zweidimensionaler Laufzeit-Konfiguration einzusetzen:
	\begin{itemize}
		\item 1 Dimension: Kategorie (z.B. Package, Layer, etc.)
		\item 2 Dimension: Log-Level
	\end{itemize}
	\item Die Log Level Konfiguration von GVS 1.0 wird nicht mehr übernommen. Stattdessen wird im Benutzerhandbuch die Verwendung der JMX-Konsole (Java Management Extensions) beschrieben. Mit dieser kann das Log Level zur Laufzeit angepasst werden.
\end{enumerate} 

\section{Ausblick}
\begin{enumerate}
	\item In der nächsten Woche steht viel Detailarbeit bei der Visualisierung des Graphen an. 
	\item Ziel bis zum ersten Release ist es, die Visualisierung des Graphen abgeschlossen und getested zu haben.
	\item Sobald die Graph Visualisierung fehlerfrei funktioniert, wird die Darstellung des Trees implementiert.
\end{enumerate}

\section{Nächster Termin}
\begin{tabu} to \linewidth {l X }
	\toprule
	Termin & 14 oder 15.11.2017  \\
	Bemerkungen & Ort wird noch bekanntgegeben   \\
	\bottomrule
\end{tabu}

\end{document}
