% !TeX spellcheck = de_CH
\documentclass[11pt, a4paper,oneside]{scrartcl}

% header
\usepackage[headsepline]{scrlayer-scrpage}
\clearpairofpagestyles
\ohead{\today}
\ihead{GVS Meeting Protokoll}
\cfoot{\pagemark}

% language
\usepackage[T1]{fontenc}
\usepackage[utf8]{inputenc}
\usepackage[ngerman]{babel}

% tables
\usepackage{tabu}
\usepackage{booktabs}
\tabulinesep=2mm

% images
\usepackage{graphicx}

% layout
\usepackage[]{geometry}
\usepackage{multicol}
\setlength{\columnsep}{1cm}


% math
\usepackage{amsmath}
\usepackage{amssymb}
\usepackage{amsfonts}
\usepackage{enumitem}



\begin{document}
	\centering
	\rule{\textwidth}{1.6pt}\vspace*{-\baselineskip}\vspace*{2pt} %Thick horizontal line.
	\rule{\textwidth}{0.4pt}\\[\baselineskip] %Thin horizontal line.
	{\LARGE [ Meeting Protokoll Woche 13 ]}\\[0.2\baselineskip] %Title.
	\rule{\textwidth}{0.4pt}\vspace*{-\baselineskip}\vspace{3.2pt} %Thin horizontal line.
	\rule{\textwidth}{1.6pt}\\[2\baselineskip] %Thick horizontal line.
	
	\begin{tabu} to \linewidth {l X }
		\toprule
		\textbf{Thema} & \textbf{Wöchentliches GVS Meeting} \\
		\midrule
		Ort & Raum 1.223 \\
		Datum & 12.12.2017  \\
		Uhrzeit &  17:10 - 18:15  \\
		Teilnehmer & 
		\begin{minipage}[t]{\textwidth}
			\begin{itemize}
				\item Murièle Trentini
				\item Michael Wieland
				\item Thomas Letsch
			\end{itemize}
		\end{minipage}
		\\
		\bottomrule
	\end{tabu}
	
	
	\section{Rückblick}
	\begin{enumerate}
		\item Die ScalablePane funktioniert nun einwandfrei
		\item Ein serverseitiger Watchdog wurde implementiert, der eine Socket Verbindung trennt, sobald der Client abgestürzt ist.
		\item Ein grosser Teil der Projektdokumentation wurde geschrieben.
	\end{enumerate}
	
	\section{Aktuelles}
	\begin{enumerate}
		\item Die offenen Punkte aus der Verbesserungs-Liste von Herr Letsch, sind im Kapitel Ausblick genauer beschrieben.
	\end{enumerate} 
	
	\section{Beschlüsse}
	\begin{enumerate}
		\item Die Anforderung zu der konfigurierbaren Schriftgrösse fällt durch die \textit{ScalableScrollPane} weg. Die maximale Schriftgrösse kann im CSS definiert werden.
		\item Der SwitchButton soll neu für die Funktion ''Force Layout'' verwendet werden. Im GVS 1.0 wurde dieser für die Unterscheidung zwischen Soft und Stable Layout verwendet. Dies hatte aber keinen praktischen Nutzen.
		\item Die Anforderung bezüglich der ''NO GVS'' Umgebungsvariable wurde von Herr Letsch zur Kenntnis genommen. Die Anforderung muss nicht mehr weiter dokumentiert werden.
		\item Die Klasse GVSTreeWithCollection wird für Forests verwendet.
		\item Das Poster fliesst nicht in die Bewertung der Studienarbeit ein.
		\item Das Projektteam setzt die Variante 2 der beiden vom Studiengang vorgegebenen Dokumentationstypen um.
		\item Es muss eine EPrint Version gemäss dem Mail von Frau Furrer erstellt werden.
	\end{enumerate}
	
	\section{Ausblick}
	\begin{enumerate}
		\item Die Dokumentation wird weiter verbessert, sowie die Dokumentenstruktur der Vorlage angepasst.
		\item Das Abstract muss bis Dienstag 19.12.2017 um 10:00 auf https://abstract.hsr.ch hochgeladen sein.
		\item Die finale DVD muss bis Freitag 22.12.2017 um 17:00 in das Fach von Herr Letsch beim Sekretariat gelegt werden.
	\end{enumerate}
	
	\section{Nächster Termin}
	\begin{tabu} to \linewidth {l X }
		\toprule
		Termin & 19.12.2017  \\
		Bemerkungen & 1.223 \\
		\bottomrule
	\end{tabu}
	
\end{document}
