% !TeX spellcheck = de_CH
\documentclass[11pt, a4paper,oneside]{scrartcl}

% header
\usepackage[headsepline]{scrlayer-scrpage}
\clearpairofpagestyles
\ohead{\today}
\ihead{GVS Meeting Protokoll}
\cfoot{\pagemark}

% language
\usepackage[T1]{fontenc}
\usepackage[utf8]{inputenc}
\usepackage[ngerman]{babel}

% tables
\usepackage{tabu}
\usepackage{booktabs}
\tabulinesep=2mm

% images
\usepackage{graphicx}

% layout
\usepackage[]{geometry}
\usepackage{multicol}
\setlength{\columnsep}{1cm}


% math
\usepackage{amsmath}
\usepackage{amssymb}
\usepackage{amsfonts}
\usepackage{enumitem}



\begin{document}
\centering
\rule{\textwidth}{1.6pt}\vspace*{-\baselineskip}\vspace*{2pt} %Thick horizontal line.
\rule{\textwidth}{0.4pt}\\[\baselineskip] %Thin horizontal line.
{\LARGE [ Meeting Protokoll Woche 6 ]}\\[0.2\baselineskip] %Title.
\rule{\textwidth}{0.4pt}\vspace*{-\baselineskip}\vspace{3.2pt} %Thin horizontal line.
\rule{\textwidth}{1.6pt}\\[2\baselineskip] %Thick horizontal line.

\begin{tabu} to \linewidth {l X }
	\toprule
	\textbf{Thema} & \textbf{Wöchentliches GVS Meeting} \\
	\midrule
	Ort & Raum 1.223 \\
	Datum & 24.10.2017  \\
	Uhrzeit &  17:15 -  \\
	Teilnehmer & 
	\begin{minipage}[t]{\textwidth}
	  	\begin{itemize}
			\item Murièle Trentini
			\item Michael Wieland
			\item Thomas Letsch
	  	\end{itemize}
	\end{minipage}
	\\
	\bottomrule
\end{tabu}


\section{Rückblick}
\begin{enumerate}
	\item Das Projektteam hat die Elaborations-Phase abgeschlossen. Die Projektdokumentation wurde auf den aktuellen Stand gebracht und an Herrn Letsch zum Review gesendet.
	\item Das Projektteam hat mit der Construction-Phase begonnen und erste neue Klassen erstellt sowie die Programmabläufe besprochen.
\end{enumerate}

\section{Aktuelles}
\begin{enumerate}
	\item Eine Implementierung des Dependency Injection Framework ''Guice'' wurde in betracht gezogen. Priorität haben jedoch die Arbeiten am Presentation Layer. Sofern sich das DI Framework ohne grösseren Aufwand integrieren lässt, könnte das parallel durchgeführt werden. Somit kann die Arbeit besser aufgeteilt werden. Vom Dependency Framework würde man besonders beim Testing profitieren. (Inject Mock)
\end{enumerate}

\section{Beschlüsse}
\begin{enumerate}
	\item Das Projektteam hat nützliches Feedback zum Projektdokument erhalten. Die Inputs werden in dieser Woche umgesetzt . 
	\item Das nächste Meeting am 01.11.2017 (Allerheiligen) wird auf den 08.11.2017 verschoben.
\end{enumerate}

\section{Ausblick}
\begin{enumerate}
	\item Priorität hat in den dieser und in der nächsten Woche die Implementierung des Presentation Layers. Nebenbei werden situativ Verbesserungen gemacht. 
\end{enumerate}

\section{Nächster Termin}
\begin{tabu} to \linewidth {l X }
	\toprule
	Termin & 08.11.2017  \\
	Bemerkungen & Ort wird noch bekanntgegeben   \\
	\bottomrule
\end{tabu}

\end{document}
