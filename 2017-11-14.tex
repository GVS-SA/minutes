% !TeX spellcheck = de_CH
\documentclass[11pt, a4paper,oneside]{scrartcl}

% header
\usepackage[headsepline]{scrlayer-scrpage}
\clearpairofpagestyles
\ohead{\today}
\ihead{GVS Meeting Protokoll}
\cfoot{\pagemark}

% language
\usepackage[T1]{fontenc}
\usepackage[utf8]{inputenc}
\usepackage[ngerman]{babel}

% tables
\usepackage{tabu}
\usepackage{booktabs}
\tabulinesep=2mm

% images
\usepackage{graphicx}

% layout
\usepackage[]{geometry}
\usepackage{multicol}
\setlength{\columnsep}{1cm}


% math
\usepackage{amsmath}
\usepackage{amssymb}
\usepackage{amsfonts}
\usepackage{enumitem}



\begin{document}
\centering
\rule{\textwidth}{1.6pt}\vspace*{-\baselineskip}\vspace*{2pt} %Thick horizontal line.
\rule{\textwidth}{0.4pt}\\[\baselineskip] %Thin horizontal line.
{\LARGE [ Meeting Protokoll Woche 9 ]}\\[0.2\baselineskip] %Title.
\rule{\textwidth}{0.4pt}\vspace*{-\baselineskip}\vspace{3.2pt} %Thin horizontal line.
\rule{\textwidth}{1.6pt}\\[2\baselineskip] %Thick horizontal line.

\begin{tabu} to \linewidth {l X }
	\toprule
	\textbf{Thema} & \textbf{Wöchentliches GVS Meeting} \\
	\midrule
	Ort & Raum 1.223 \\
	Datum & 14.11.2017  \\
	Uhrzeit &  17:10 - 17:50 \\
	Teilnehmer & 
	\begin{minipage}[t]{\textwidth}
	  	\begin{itemize}
			\item Murièle Trentini
			\item Michael Wieland
			\item Thomas Letsch
	  	\end{itemize}
	\end{minipage}
	\\
	\bottomrule
\end{tabu}


\section{Rückblick}
\begin{enumerate}
	\item Erste Verbesserungen an der Graph Visualisierung wurden umgesetzt. (Vertex und Edge Styling, Vertex Icon Support, Drag Support verbessert)
	\item Im Access und Business Layer wurden die nebenläufigen Klassen korrekt synchronisiert
	\item Die Replay Funktionalität kann nun gestoppt und abgebrochen werden.
\end{enumerate}

\section{Aktuelles}
\begin{enumerate}
	\item Das aktuelle Meeting wurde auf den 14.11.2017 vor verschoben (ursprünglich geplant: 15.11.2017).
	\item Das Projektteam hat die unterschriebenen Urheber und Nutzungsbedingungen an Herr Letsch retourniert.
	\item Die Visualisierung von gerichteten Graphen ist wesentlich komplizierter als erwartet und wird mehr Zeit als geplant in Anspruch nehmen.
\end{enumerate}

\section{Beschlüsse}
\begin{enumerate}
	\item Unit Tests werden wann sinnvoll geschrieben. Höhere Priorität haben die Test Programme, die einen Client Aufruf simulieren (Integrationstest). In der aktuellen Phase würden die Unit Tests nur sehr kurz leben, da noch viel refactored wird.
	\item Im Benutzerhandbuch soll die Verwendung der FontAwesome Icons beschrieben werden. Ebenfalls soll der Hinweis dokumentiert werden, dass FontAwesome ab Java 9 in JavaFX integriert ist.
	\item Der Abgabetermin für den ersten Release ist am Mittwoch Morgen 22.11.2017. Alle nötigen Informationen für die Inbetriebnahme sollen minimal beschrieben werden (z.B Textdokument). Ebenfalls sollen bekannte Bugs beschrieben werden.
\end{enumerate} 

\section{Ausblick}
\begin{enumerate}
	\item In der nächsten Woche wird die Funktionsweise des Layouters verbessert (Kopieren und übernehmen von vorgängig berechneten Koordinaten, Keine Neuberechnungen für gedraggte Nodes)
 	\item Die Visualisierung von gerichteten Kanten wird in den nächsten Tagen abgeschlossen.
 	\item Abschliessende Tests für die Graph Visualisierung und erste Schritte in Richtung Visualisuerung der Trees.
 	\item In der nächsten Woche steht der erste Release an.
\end{enumerate}

\section{Nächster Termin}
\begin{tabu} to \linewidth {l X }
	\toprule
	Termin & 22.11.2017  \\
	Bemerkungen & Ort wird noch bekanntgegeben   \\
	\bottomrule
\end{tabu}

\end{document}
