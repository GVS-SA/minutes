% !TeX spellcheck = de_CH
\documentclass[11pt, a4paper,oneside]{scrartcl}

% header
\usepackage[headsepline]{scrlayer-scrpage}
\clearpairofpagestyles
\ohead{\today}
\ihead{GVS Meeting Protokoll}
\cfoot{\pagemark}

% language
\usepackage[T1]{fontenc}
\usepackage[utf8]{inputenc}
\usepackage[ngerman]{babel}

% tables
\usepackage{tabu}
\usepackage{booktabs}
\tabulinesep=2mm

% images
\usepackage{graphicx}

% layout
\usepackage[]{geometry}
\usepackage{multicol}
\setlength{\columnsep}{1cm}


% math
\usepackage{amsmath}
\usepackage{amssymb}
\usepackage{amsfonts}
\usepackage{enumitem}



\begin{document}
	\centering
	\rule{\textwidth}{1.6pt}\vspace*{-\baselineskip}\vspace*{2pt} %Thick horizontal line.
	\rule{\textwidth}{0.4pt}\\[\baselineskip] %Thin horizontal line.
	{\LARGE [ Meeting Protokoll Woche 10 ]}\\[0.2\baselineskip] %Title.
	\rule{\textwidth}{0.4pt}\vspace*{-\baselineskip}\vspace{3.2pt} %Thin horizontal line.
	\rule{\textwidth}{1.6pt}\\[2\baselineskip] %Thick horizontal line.
	
	\begin{tabu} to \linewidth {l X }
		\toprule
		\textbf{Thema} & \textbf{Wöchentliches GVS Meeting} \\
		\midrule
		Ort & Raum 5.207 \\
		Datum & 22.11.2017  \\
		Uhrzeit &  17:15 - 18:25 \\
		Teilnehmer & 
		\begin{minipage}[t]{\textwidth}
			\begin{itemize}
				\item Murièle Trentini
				\item Michael Wieland
				\item Thomas Letsch
			\end{itemize}
		\end{minipage}
		\\
		\bottomrule
	\end{tabu}
	
	
	\section{Rückblick}
	\begin{enumerate}
		\item Der erste Release konnte erfolgreich abgeschlossen werden. Das Projektteam hat den aktuellen Stand released und die Artefakte zusammen mit einer Bugliste an Herr Letsch gesendet.
		\item Der Tree Layouter wurde komplett überarbeitet (Reingold-Rilford-Algorithmus für n-Trees)
		\item Der aktuelle Stand der Software wurde umfassend getestet. Die gefunden Fehler wurden entweder behoben oder als Bug Issue im Jira erfasst.
		\item Das UI Styling wurde umfassend überarbeitet und auf Usability getestet.
	\end{enumerate}
	
	\section{Aktuelles}
	\begin{enumerate}
		\item Das Log Level kann auf Package Ebene mittels JMX angepasst werden. Der Vorgang wird in der Benutzeranleitung genauer beschrieben.
		\item Als Ersatz für den Cluster Splitter wird ein neuer Algorithmus für eine ansprechende Darstellung von B-Trees verwendet.
	\end{enumerate}
	
	\section{Beschlüsse}
	\begin{enumerate}
		\item Das Projektteam hat nützliches Feedback zum Release 1.0 von Herr Letsch erhalten.
		\begin{enumerate}
			\item Beim Speichern von Sessions muss die *.gvs File Extension sichergestellt werden.
			\item Der Speed Slider soll während dem Replay disabled werden. 
			\item Die Graphen sollen zentriert angezeigt werden. Ebenfalls soll der Inhalt stets korrekt skaliert werden.
		\end{enumerate}
		\item Eine Sammlung an nützlichen Testprogrammen wird in einem neuen, versionierten Repository angelegt.		
		\item Das Anlegen des Communication XML File durch den Socket Server ist überflüssig und darf entfernt werden.
	\end{enumerate} 
	
	\section{Ausblick}
	\begin{enumerate}
		\item In der nächsten Woche werden die beiden Client Libraries um Generics erweitert.
		\item Ebenfalls wird der Layout Algorithmus für Binary Trees implementiert.
	\end{enumerate}
	
	\section{Nächster Termin}
	\begin{tabu} to \linewidth {l X }
		\toprule
		Termin & 28.11.2017  \\
		Bemerkungen & Ort wird noch bekanntgegeben  \\
		\bottomrule
	\end{tabu}
	
\end{document}
